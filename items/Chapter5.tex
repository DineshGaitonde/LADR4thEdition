\section*{Chapter 5: Eigenvalues, Eigenvectors, and Invariant Subspaces}

%%%%%%%%%%%%%%%%%%%%%%%%%%%%%%%%%%%%%%%%
%% Invariant Subspaces
%%%%%%%%%%%%%%%%%%%%%%%%%%%%%%%%%%%%%%%%
\subsection*{Exercises 5A - Invariant Subspaces}


\begin{problem}{1}
Suppose $T\in\Hom(V)$ and $U$ is a subspace of $V$.
\begin{enumerate}[(a)]
\item Prove that if $U\subseteq \Null T$, then $U$ is invariant under $T$.
\item Prove that if $\Range T\subseteq U$, then $U$ is invariant under $T$.
\end{enumerate}
\end{problem}
\begin{proof}

\begin{enumerate}[(a)]
\item Consider $u \in U$. $Tu = 0$. $0 \in U$. So $T(u) \in U$. Hence, $U$ is invariant under $T$.
\item Consider $u \in U$. $T(u) \in range\ T \subseteq U$. Hence $T(u) \in U$. Hence $U$ is invariant under $T$.  \qedhere
\end{enumerate}


\end{proof}


\begin{problem}{2}
Suppose $T\in\Hom(V)$ and $V_1, \ldots V_m$ are invariant subspaces of $T$. Prove $V_1 + \ldots + V_m$ is invariant under $T$.
\end{problem}
%%%%%%%%%%%%%%%%%%%%%%%%%%%%%%%%%%%%%%%%
%%%%%%%%%%%%%%%%%%%%%%%%%%%%%%%%%%%%%%%%