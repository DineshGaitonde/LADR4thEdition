\section*{Chapter 5: Eigenvalues, Eigenvectors, and Invariant Subspaces}

%%%%%%%%%%%%%%%%%%%%%%%%%%%%%%%%%%%%%%%%
%% Invariant Subspaces
%%%%%%%%%%%%%%%%%%%%%%%%%%%%%%%%%%%%%%%%
\subsection*{Exercises 5A - Invariant Subspaces}

%%%%% Problem 1
\begin{problem}{1}
Suppose $T\in\Hom(V)$ and $U$ is a subspace of $V$.
\begin{enumerate}[(a)]
\item Prove that if $U\subseteq \Null T$, then $U$ is invariant under $T$.
\item Prove that if $\Range T\subseteq U$, then $U$ is invariant under $T$.
\end{enumerate}
\end{problem}
\begin{proof}

\begin{enumerate}[(a)]
\item Consider $u \in U$. $Tu = 0$. $0 \in U$. So $T(u) \in U$. Hence, $U$ is invariant under $T$.
\item Consider $u \in U$. $T(u) \in range\ T \subseteq U$. Hence $T(u) \in U$. Hence $U$ is invariant under $T$.  \qedhere
\end{enumerate}
\end{proof}
%%%%%%%%%%%%%%%

%%%%% Problem 2
\begin{problem}{2}
Suppose $T\in\Hom(V)$ and $V_1, \ldots V_m$ are invariant subspaces of $T$. Prove $V_1 + \ldots + V_m$ is invariant under $T$.
\end{problem}
\begin{proof}
Consider an arbitrary element $v \in V_1 + \ldots + V_m$. It can be expressed as $v = v_1 + \ldots + v_m$, where $v_i \in V_i$.
$T(v) = T(v_1 + \ldots + v_m) = T(v_1) + \ldots + T(v_m)$. Since $V_i$ is invariant under $T$, $T(v_i) \in V_i$. 
Hence $T(v) \in V_1 + \ldots + V_m$ if $v \in V_1 + \ldots + V_m$.
\end{proof}
%%%%%%%%%%%%%%%

%%%%% Problem 3
\begin{problem}{3}
Suppose $T\in\Hom(V)$. Prove that the intersection of every collection of subspaces of $V$ invariant under $T$ is invariant 
under $T$.
\end{problem}
\begin{proof}
Consider an arbitrary element $v$ in the intersection of the invariant subspaces ($(V_1, V_2, \ldots)$). Since each of the 
subspaces are invariant, $T(v) \in V_i$. Hence $T(v) \in V_1 \cap V_2 \cap \ldots$. Hence intersection of invariant subspaces
under some linear transformation is invariant.
\end{proof}
%%%%%%%%%%%%%%%

%%%%% Problem 4
\begin{problem}{4}
Prove or give a counterexample: If $V$ is finite-dimensional and $U$ is a subspace of $V$ that is invariant under every
operator on $V$, then $U = \{0\}$ or $U = V$.
\end{problem}
\begin{proof}
Suppose to the contrary that $U$ is neither the trivial subspace, nor the full space. Since $U \neq \{0\}$, there is at 
least one basis vector in $U$. Since $U \neq V$, there exists at least one vector $v \notin U$. Consider any operator that 
maps $u$ to $v$. Obviously this map does not keep $U$ invariant. 
\end{proof}
%%%%%%%%%%%%%%%

%%%%% Problem 5
\begin{problem}{5}
Suppose $T\in\Hom(\mathbb{R}^2)$ defined by $T(x, y) = (-3y, x)$. Find the eigenvalues of $T$.
\end{problem}
\begin{proof}
Suppose to the contrary that $U$ is neither the trivial subspace, nor the full space. Since $U \neq \{0\}$, there is at 
least one basis vector in $U$. Since $U \neq V$, there exists at least one vector $v \notin U$. Consider any operator that 
maps $u$ to $v$. Obviously this map does not keep $U$ invariant. 
\end{proof}
%%%%%%%%%%%%%%%