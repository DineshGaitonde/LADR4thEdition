\section*{Chapter 5}

\subsection*{Exercises 5A}

\begin{problem}{1}
Suppose $T \in \mathcal{L}(V)$ and $U$ is a subspace of $V$.
\end{problem}

(a) Prove that if $U \subseteq null\ T$, then $U$ is invariant under $T$.

Consider $u \in U$. $Tu = 0$. $0 \in U$. So $T(u) \in U$. Hence, $U$ is invariant under $T$.

(b) Prove that if $range\ T \subseteq U$, then $U$ is invariant under $T$.

Consider $u \in U$. $T(u) \in range\ T \subseteq U$. Hence $T(u) \in U$. Hence $U$ is invariant under $T$.


\begin{problem}{2}

\end{problem}
